\documentclass{article}
\usepackage[utf8]{inputenc}
\usepackage{listings}
\usepackage[a4paper, total={19cm}]{geometry}

\begin{document}
\section{Fahrenheit para Celsius}
Foi escrito programa que aplica a fórmula de conversão de temperaturas em Fahrenheit ($F$) para seu valor equivalente na escala Celsius ($C$):

\[C = \frac{(F-32)}{1.8}\]

As temperaturas iniciais são obtidas em um loop DO, partindo-se de 0 até 100 com intervalos de 10. O output gerado, então escrito em arquivo .dat foi:

\begin{lstlisting}

   0.00000000      -17.7777786      -16.6666679      -6.24999739E-02
   10.0000000      -12.2222223      -11.1111116      -9.09090564E-02
   20.0000000      -6.66666698      -5.55555582     -0.166666672    
   30.0000000      -1.11111116       0.00000000      -1.00000000    
   40.0000000       4.44444466       5.55555582      0.250000000    
   50.0000000       10.0000000       11.1111116      0.111111164    
   60.0000000       15.5555563       16.6666679       7.14286044E-02
   70.0000000       21.1111126       22.2222233       5.26315570E-02
   80.0000000       26.6666679       27.7777786       4.16666493E-02
   90.0000000       32.2222214       33.3333359       3.44828665E-02
   100.000000       37.7777786       38.8888893       2.94117536E-02


\end{lstlisting}
 
A penúltima e a última coluna, respectivamente, representam a aproximação dada pela fórmula

\[C = \frac{(F-30)}{2}\]

e a diferença relativa entre os valores precisos (segunda coluna) e os aproximados (terceira coluna).

\section{Fatoriais e a aproximação de Stirling}

O programa escrito faz uso de função recursiva para multiplicar um número natural pelos seus antecessores, retornando o fatorial. Em um loop DO, os fatoriais dos números de um a vinte, escritos na primeira coluna, são então calculados e direcionados à segunda coluna de um arquivo .dat. \par
Imprimiu-se também a aproximação de Stirling na terceira coluna e sua diferença relativa ao resultados exatos (coluna dois). \par
A saída do programa assim se constituiu:

\begin{lstlisting}
           1                    1  0.92213704972951738        7.7862950270482623E-002
           2                    2   1.9190044947416793        4.0497752629160333E-002
           3                    6   5.8362102042491442        2.7298299291809307E-002
           4                   24   23.506178315281900        2.0575903529920819E-002
           5                  120   118.01918751963078        1.6506770669743532E-002
           6                  720   710.07832390339013        1.3780105689735932E-002
           7                 5040   4980.3969596163970        1.1826000076111707E-002
           8                40320   39902.405701853524        1.0357001442124899E-002
           9               362880   359536.97610976401        9.2124776516644205E-003
          10              3628800   3598696.7616594764        8.2956454862553909E-003
          11             39916800   39615638.835241772        7.5447221410089929E-003
          12            479001600   475687666.43874627        6.9184185632234349E-003
          13           6227020800   6187242003.8929253        6.3880943045950224E-003
          14          87178291200   86661039790.316376        5.9332593305479273E-003
          15        1307674368000   1300431332662.9717        5.5388677137612292E-003
          16       20922789888000   20814124818110.262        5.1936223836029508E-003
          17      355687428096000   353948516318011.12        4.8888761328936903E-003
          18     6402373705728000   6372808198376919.0        4.6178977844779168E-003
          19   121645100408832000   1.2111285815823707E+017   4.3753694049833243E-003
          20  2432902008176640000   2.4227883519687260E+018   4.1570339347509330E-003
\end{lstlisting}
A diminuição da diferença entre o valor real e a aproximação de Stirling conforme se aumenta o valor de N é condizente com o caráter das aproximações assintóticas de ``N-grande'', especialmente úteis para o cálculo de termos com posições mais avançadas em uma sequência.
\section{Série de Taylor para o seno}

Criou-se uma função SEN que leva dois argumentos: o valor para o qual deseja-se aproximar o seno (N) e a casa decimal (PREC) até a qual a aproximação deve bater com o resultado da função SIN, nativa.
Foi executado um loop DO WHILE, calculando e somando-se os termos do polinômio de Taylor até que a precisão desejada seja inserida:
\[|SEN(N) - SIN(N)| < 10^{(-PREC)}\]
Os termos ímpares foram obtidos com a expressão I*2+1, na qual I é a variável de iteração, acrescida uma unidade em cada ciclo. A alternância de sinal foi alcançada com uso de (-1) ** (I).
A função então imprime em arquivo .dat os dados relacionados à aproximação, incluindo sua ordem (grau do polinômio de Taylor) necessária para se alcançar a precisão inserida, e a diferença entre as funções SIN e SEN para um mesmo valor de N (desvio). Chamar SEN para 5 números diferentes, adotando-se PREC = 6, resulta, no arquivo de saída:

\begin{lstlisting}
no output :(
\end{lstlisting}

\section{Vetores no plano}

Criou-se uma função que recebe dois reais, as coordenadas cartesianas de um vetor bidimensional, e imprime as coordenadas polares do mesmo vetor. Para calcular-se o raio r aplicou-se a raiz da soma dos quadrados das coordenadas cartesianas, e obteve-se $\theta$ com o uso da função nativa para arcotangente (ATAN), aplicada sobre a razão entre a coordenada y e a x ($\frac{y}{x}$).\par
A partir das coordenadas polares, pode-se facilmente aplicar a rotação anti-horária de um ângulo $\phi$, simplesmente somando-se seu valor ao de $\theta$. Para obter as coordenadas cartesianas do vetor apos o giro, atribuem-se os valores de $r\cos{(\theta + \phi)}$ e $r\sin{(\theta + \phi)}$, respectivamente, para x e y.

OUTPUUUUUUUTTSSSS

\section{Organize uma lista}

Foi criada inicialmente uma subrotina que recebe dois argumentos: uma array (ARR) e um inteiro (POS) que representa uma posição em ARR. Seu papel é deslocar todos os elementos, a partir de POS, uma posição à frente no vetor, de forma que o a última coordenada é sempre perdida. A cada input recebido, um loop DO percorre ARR de trás pra frente, até que encontre um número N menor que o input, ou que chegue ao início de ARR. Neste ponto, a subrotina desloca os números à frente de N e o programa atribui o input ao local da array logo após N, ou desloca todos os números e atribui o input à primeira posição, no caso de ser menos que todos os elementos de ARR).

OUTOUTOTUOTTUUT

\section{Valores médios e desvio padrão}

Elaborou-se um programa que aplica as fórmulas de desvio padrão, média aritmética e média geométrica a um conjunto de dados.
Utilizou-se o seguinte conjunto de números gerados aleatóriamente:

\begin{lstlisting}
 90 21 67 82 57 47 10 60 18 74 45 99 62  9 61 17 81 65 13 85 70 43 86 87 29
 33 42 26 94 58 89 95 12 30 84 69  6 77  2 39 23 73 31 53 36 44 24 27 55 80
 22 66  1 25 28 91  7 19 51 41 76 20 83  3 46 98 52 35 78 32 34 88 49 15 38
 64 59 97 93  5 75 37 54 14 71 96 79 68 16 63 72 56 40 92 50 100 48  4  8 11
\end{lstlisting}

De forma que se obtiveram os seguintes resultados:

\begin{lstlisting}
MED. ARITMÉTICA   50.500000000000007     
MED. GEOMÉTRICA   37.992689344834297     
DESVIO PADRÃO     28.866070047722072     
\end{lstlisting}

A média geométrica é especialmente útil, por exemplo, na avaliação de crescimentos proporcionais. Supondo que uma população inicialmente  de 100 pessoas passe a 190 em uma década, 304 na próxima, então para 456 na seguinte, correspondendo a incrementos proporcionais de 90\%, 60\% e 50\%. A média aritmética dos crescimentos resultaria então no valor 66,66\%, que se aplicado três vezes à população inicial resulta em aproximadamente 463 pessoas. Já a média geométrica $\sqrt[3]{1.9 \times 1.6 \times 1.5}$ (aproximadamente 1.6583) trivialmente coincide com o resultado real se usada ao cubo e multiplicando o valor inicial para a população. A média aritmética, nesses casos, exagera a taxa de crescimento esperada.
\end{document}