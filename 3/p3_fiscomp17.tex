%% ================================================================================
%% This LaTeX file was created by AbiWord.                                         
%% AbiWord is a free, Open Source word processor.                                  
%% More information about AbiWord is available at http://www.abisource.com/        
%% ================================================================================

\documentclass[a4paper,portrait,12pt]{article}
\usepackage[latin1]{inputenc}
\usepackage{calc}
\usepackage{setspace}
\usepackage{fixltx2e}
\usepackage{graphicx}
\usepackage{multicol}
\usepackage[normalem]{ulem}
%% Please revise the following command, if your babel
%% package does not support en-US
\usepackage[en]{babel}
\usepackage{color}
\usepackage{hyperref}
 
\begin{document}


\begin{flushleft}
Introdu\c{c}\~{a}o \`{a} F\'{i}sica Computacional 1S/2017
\end{flushleft}


\begin{flushleft}
Projeto 3 --- Derivadas e integrais num\'{e}ricas
\end{flushleft}


\begin{flushleft}
In\'{i}cio: 07 de Abril de 2017
\end{flushleft}


\begin{flushleft}
Data da entrega do relat\'{o}rio: 28 de Abril de 2017
\end{flushleft}





\begin{flushleft}
Prof.: Eric C. Andrade
\end{flushleft}





\begin{flushleft}
Data m\'{a}xima de entrega do relat\'{o}rio: 05 de Maio de 2017
\end{flushleft}





\begin{flushleft}
Descri\c{c}\~{a}o:
\end{flushleft}


\begin{flushleft}
Discutiremos aqui brevemente alguns conceitos b\'{a}sicos de c\'{a}lculo num\'{e}rico que s\~{a}o muito \'{u}teis no dia a dia da
\end{flushleft}


\begin{flushleft}
pesquisa. O ingrediente b\'{a}sico que permeia todos esses m\'{e}todos \'{e} a expans\~{a}o em s\'{e}rie de Taylor de uma fun\c{c}\~{a}o f (x)
\end{flushleft}


\begin{flushleft}
ao redor de x0
\end{flushleft}


\begin{flushleft}
f (x) = f (x0 ) + f 0 (x0 ) (x $-$ x0 ) +
\end{flushleft}





1 00


1


2


3


\begin{flushleft}
f (x0 ) (x $-$ x0 ) + f 000 (x0 ) (x $-$ x0 ) + . . .
\end{flushleft}


2!


3!





(1)





\begin{flushleft}
Sugest\~{a}o de execu\c{c}\~{a}o:
\end{flushleft}


\begin{flushleft}
Aula 5
\end{flushleft}





\begin{flushleft}
1) Derivada num\'{e}rica
\end{flushleft}


\begin{flushleft}
Dada uma fun\c{c}\~{a}o f (x), podemos obter uma aproxima\c{c}\~{a}o para sua derivada utilizando a defini\c{c}\~{a}o
\end{flushleft}


\begin{flushleft}
f 0 (x) = limh$\rightarrow$0
\end{flushleft}





\begin{flushleft}
f (x + h) $-$ f (x)
\end{flushleft}


,


\begin{flushleft}
h
\end{flushleft}





(2)





\begin{flushleft}
s\'{o} que para um h finito
\end{flushleft}


\begin{flushleft}
f 0 (x) $\approx$
\end{flushleft}





\begin{flushleft}
f (x + h) $-$ f (x)
\end{flushleft}


.


\begin{flushleft}
h
\end{flushleft}





(3)





\begin{flushleft}
\`{A} primeira vista, parece que tudo que temos que fazer \'{e} escolher um valor de h bem pequeno que obteremos o
\end{flushleft}


\begin{flushleft}
valor exato da derivada. Como veremos explicitamente em um exemplo, esse n\~{a}o \'{e} caso, pois rapidamente perdemos
\end{flushleft}


\begin{flushleft}
precis\~{a}o num\'{e}rica.
\end{flushleft}


\begin{flushleft}
A primeira pergunta que nos fazemos \'{e} qual \'{e} o erro que comentemos ao passar de (2) para (3)? Para respondermos
\end{flushleft}


\begin{flushleft}
essa quest\~{a}o, escrevemos
\end{flushleft}


\begin{flushleft}
f (x + h) = f (x) + hf 0 (x) +
\end{flushleft}





\begin{flushleft}
h2 00
\end{flushleft}


\begin{flushleft}
f (x) + . . . ,
\end{flushleft}


2!





\begin{flushleft}
donde obtemos a derivada para frente de 2 pontos
\end{flushleft}


\begin{flushleft}
ff0 (x) =
\end{flushleft}





\begin{flushleft}
f (x + h) $-$ f (x)
\end{flushleft}


\begin{flushleft}
+ O (h) .
\end{flushleft}


\begin{flushleft}
h
\end{flushleft}





(4)





\begin{flushleft}
Aqui o s\'{i}mbolo O (h) quer dizer que o erro que cometemos ao truncar a expans\~{a}o em s\'{e}rie de Taylor para o c\'{a}lculo
\end{flushleft}


\begin{flushleft}
de f 0 (x) \'{e} da ordem de h. Analogamente, podemos calcular a derivada para tr\'{a}s de 2 pontos
\end{flushleft}


\begin{flushleft}
ft0 (x) =
\end{flushleft}





\begin{flushleft}
f (x) $-$ f (x $-$ h)
\end{flushleft}


\begin{flushleft}
+ O (h) .
\end{flushleft}


\begin{flushleft}
h
\end{flushleft}





(5)





\begin{flushleft}
Se combinarmos as equa\c{c}\~{o}es (4) e (5) podemos escrever a derivada sim\'{e}trica de 3 pontos
\end{flushleft}


?


1 0


\begin{flushleft}
ff (x) + ft0 (x) ,
\end{flushleft}


2


?


\begin{flushleft}
f (x + h) $-$ f (x $-$ h)
\end{flushleft}


0


\begin{flushleft}
f3s (x) =
\end{flushleft}


\begin{flushleft}
+ O h2 .
\end{flushleft}


\begin{flushleft}
2h
\end{flushleft}


0


\begin{flushleft}
f3s
\end{flushleft}


\begin{flushleft}
(x) =
\end{flushleft}





(6)





\newpage
2


\begin{flushleft}
ff0 (1)
\end{flushleft}





\begin{flushleft}
h
\end{flushleft}





\begin{flushleft}
ft0 (1)
\end{flushleft}





0


\begin{flushleft}
f3s
\end{flushleft}


(1)





0


\begin{flushleft}
f5s
\end{flushleft}


(1)





00


\begin{flushleft}
f3s
\end{flushleft}


(1)





00


\begin{flushleft}
f5s
\end{flushleft}


(1)





5 × 10$-$1


1 × 10$-$1


5 × 10$-$2


1 × 10$-$2


5 × 10$-$3


1 × 10$-$3


5 × 10$-$4


1 × 10$-$4


5 × 10$-$5


1 × 10$-$5


5 × 10$-$6


1 × 10$-$6


1 × 10$-$7


1 × 10$-$8


\begin{flushleft}
Tabela I: Derivadas num\'{e}ricas de f (x) em (10) no ponto x = 1 por meio de diferentes aproxima\c{c}\~{o}es em fun\c{c}\~{a}o do passo h.
\end{flushleft}





?


\begin{flushleft}
Note que um h\'{a} um nesse caso o erro passa agora para O h2 , ou seja melhoramos nossa converg\^{e}ncia. Isso ocorre
\end{flushleft}


\begin{flushleft}
porque o termo envolvendo a derivada segunda na expans\~{a}o em Taylor \'{e} sim\'{e}trico em h e assim a primeira corre\c{c}\~{a}o
\end{flushleft}


\begin{flushleft}
vem apenas do termo c\'{u}bico. Qual \'{e} o pre\c{c}o que pagamos para esse ganho em precis\~{a}o? Temos que agora avaliar
\end{flushleft}


\begin{flushleft}
tanto f (x + h) quanto f (x $-$ h), o que aumenta o custo computacional. Podemos continuar com esse procedimento
\end{flushleft}


\begin{flushleft}
e calcular agora da derivada sim\'{e}trica de 5 pontos
\end{flushleft}


0


\begin{flushleft}
f5s
\end{flushleft}


\begin{flushleft}
(x) =
\end{flushleft}





?


\begin{flushleft}
f (x $-$ 2h) $-$ 8f (x $-$ h) + 8f (x + h) $-$ f (x + 2h)
\end{flushleft}


\begin{flushleft}
+ O h4 ,
\end{flushleft}


\begin{flushleft}
12h
\end{flushleft}





(7)





\begin{flushleft}
que agora possui um erro da ordem de h4 com o pre\c{c}o de termos que avaliar a fun\c{c}\~{a}o tamb\'{e}m em x$\pm$2h. Naturalmente,
\end{flushleft}


\begin{flushleft}
derivadas sim\'{e}tricas envolvendo mais pontos podem ser constru\'{i}das de modo an\'{a}logo.
\end{flushleft}


\begin{flushleft}
Podemos aplicar a mesma ideia calcularmos numericamente a derivada segunda de uma fun\c{c}\~{a}o. Partimos da
\end{flushleft}


\begin{flushleft}
seguinte igualdade
\end{flushleft}


?


\begin{flushleft}
f (x + h) $-$ 2f (x) + f (x $-$ h) = f 00 (x) h2 + O h4
\end{flushleft}


\begin{flushleft}
e chegamos assim \`{a} derivada segunda sim\'{e}trica de tr\^{e}s pontos
\end{flushleft}


00


\begin{flushleft}
f3s
\end{flushleft}


\begin{flushleft}
(x) =
\end{flushleft}





?


\begin{flushleft}
f (x + h) $-$ 2f (x) + f (x $-$ h)
\end{flushleft}


\begin{flushleft}
+ O h2 .
\end{flushleft}


2


\begin{flushleft}
h
\end{flushleft}





(8)





\begin{flushleft}
Tamb\'{e}m podemos escrever uma express\~{a}o para a derivada segunda sim\'{e}trica de cinco pontos
\end{flushleft}


00


\begin{flushleft}
f5s
\end{flushleft}


\begin{flushleft}
(x) =
\end{flushleft}





?


\begin{flushleft}
$-$f (x $-$ 2h) + 16f (x $-$ h) $-$ 30f (x) + 16f (x + h) $-$ f (x + 2h)
\end{flushleft}


\begin{flushleft}
+ O h4 .
\end{flushleft}


2


\begin{flushleft}
12h
\end{flushleft}





(9)





\begin{flushleft}
Baseado nessa discuss\~{a}o, considere agora a seguinte fun\c{c}\~{a}o
\end{flushleft}


\begin{flushleft}
f (x) = e2x senx.
\end{flushleft}





(10)





\begin{flushleft}
(a) Escreva um programa FORTRAN que calcule as derivadas de f (x) para as diferentes aproxima\c{c}\~{o}es e valores
\end{flushleft}


\begin{flushleft}
de h no ponto x = 1.
\end{flushleft}


\begin{flushleft}
(b) Preencha a tabela I com o valor absoluto dos desvios em rela\c{c}\~{a}o aos resultados exatos |$\epsilon$|. Aponte o valor \'{o}timo
\end{flushleft}


\begin{flushleft}
de h em cada um dos casos e discuta seus resultados.
\end{flushleft}


0


00


0


\begin{flushleft}
(1), f5s
\end{flushleft}


\begin{flushleft}
(1) e f3s
\end{flushleft}


\begin{flushleft}
(1). Verifique que a ordem da converg\^{e}ncia
\end{flushleft}


\begin{flushleft}
(c) Fa\c{c}a um gr\'{a}fico de log10 |$\epsilon$| × log10 h para ff0 (1), f3s
\end{flushleft}


\begin{flushleft}
das aproxima\c{c}\~{o}es coincide com aquela esperada teoricamente e discuta seus resultados.
\end{flushleft}





\newpage
3





\begin{flushleft}
Figura 1: Representa\c{c}\~{a}o geom\'{e}trica do c\'{a}lculo da integral. Dividimos o intervalo [a, b] em N subintervalos de igual tamanho
\end{flushleft}


\begin{flushleft}
h = (b $-$ a) /N .
\end{flushleft}





\begin{flushleft}
2) Integra\c{c}\~{a}o num\'{e}rica
\end{flushleft}


\begin{flushleft}
A integral
\end{flushleft}


ˆ


\begin{flushleft}
I=
\end{flushleft}





\begin{flushleft}
b
\end{flushleft}





\begin{flushleft}
f (x) dx,
\end{flushleft}





(11)





\begin{flushleft}
a
\end{flushleft}





\begin{flushleft}
possui um significado geom\'{e}trico muito simples: ela d\'{a} a \'{a}rea contida sob a curva descrita pela fun\c{c}\~{a}o f (x) indo
\end{flushleft}


\begin{flushleft}
de x = a at\'{e} x = b. A ideia b\'{a}sica por tr\'{a}s dos m\'{e}todos b\'{a}sicos de integra\c{c}\~{a}o \'{e} dividir o intervalo [a, b] em N
\end{flushleft}


\begin{flushleft}
subintervalos de igual tamanho h = (b $-$ a) /N , de tal forma que a integral \'{e} agora dada por, veja figura 1
\end{flushleft}


\begin{flushleft}
ˆ b
\end{flushleft}


\begin{flushleft}
ˆ a+h
\end{flushleft}


\begin{flushleft}
ˆ a+2h
\end{flushleft}


\begin{flushleft}
ˆ b
\end{flushleft}


\begin{flushleft}
I=
\end{flushleft}


\begin{flushleft}
f (x) dx =
\end{flushleft}


\begin{flushleft}
f (x) dx +
\end{flushleft}


\begin{flushleft}
f (x) dx + . . .
\end{flushleft}


\begin{flushleft}
f (x) dx.
\end{flushleft}


(12)


\begin{flushleft}
a
\end{flushleft}





\begin{flushleft}
a
\end{flushleft}





\begin{flushleft}
a+h
\end{flushleft}





\begin{flushleft}
b$-$h
\end{flushleft}





\begin{flushleft}
O que fazemos agora \'{e} aproximarmos f (x) em cada um dos intervalos por fun\c{c}\~{o}es simples, para as quais podemos
\end{flushleft}


\begin{flushleft}
fazer a integral analiticamente. Come\c{c}amos com uma aproxima\c{c}\~{a}o linear para f (x) ao redor de x = u
\end{flushleft}


\begin{flushleft}
f (x) $\approx$ f (u) + f 0 (u) (x $-$ u) $\approx$ f (u) +
\end{flushleft}





\begin{flushleft}
f (u + h) $-$ f (u)
\end{flushleft}


\begin{flushleft}
(x $-$ u) ,
\end{flushleft}


\begin{flushleft}
h
\end{flushleft}





\begin{flushleft}
onde utilizamos a Eq. 4. Temos assim que
\end{flushleft}


?


\begin{flushleft}
ˆ u+h
\end{flushleft}


\begin{flushleft}
ˆ u+h ?
\end{flushleft}


\begin{flushleft}
f (u + h) $-$ f (u)
\end{flushleft}


1


\begin{flushleft}
I=
\end{flushleft}


\begin{flushleft}
f (x) dx =
\end{flushleft}


\begin{flushleft}
f (u) +
\end{flushleft}


\begin{flushleft}
(x $-$ u) dx = h (f (u + h) + f (u)) ,
\end{flushleft}


(13)


\begin{flushleft}
h
\end{flushleft}


2


\begin{flushleft}
u
\end{flushleft}


\begin{flushleft}
u
\end{flushleft}


?


?


\begin{flushleft}
que \'{e} conhecida como regra do trap\'{e}zio. Seu erro local \'{e} O h3 e o erro global \'{e} de O h2 .1 A integral no intervalo
\end{flushleft}





1





\begin{flushleft}
O erro local \'{e} o erro em cada uma das N parti\c{c}\~{o}es. Ele depende de h = (b $-$ a) /N . Como temos N parti\c{c}\~{o}es, temos que somar N erros
\end{flushleft}


\begin{flushleft}
locais e, por isso, o erro global \'{e} sempre uma pot\^{e}ncia de h menor do que o erro local.
\end{flushleft}





\newpage
4


\begin{flushleft}
h
\end{flushleft}





\begin{flushleft}
Regra do trap\'{e}zio Regra de Simpson
\end{flushleft}





1/2


1/4


1/8


1/16


1/32


1/64


1/128


1/256


1/1024


1/2048


1/4096


1/8192


\begin{flushleft}
Tabela II: Integral num\'{e}rica de f (x) em (16) no intervalo [0, 1] por meio de duas diferentes aproxima\c{c}\~{o}es como fun\c{c}\~{a}o da
\end{flushleft}


\begin{flushleft}
parti\c{c}\~{a}o do intervalo h.
\end{flushleft}





\begin{flushleft}
[a, b] \'{e} ent\~{a}o dada por
\end{flushleft}


\begin{flushleft}
IT $\approx$ h [0.5f (a) + f (a + h) + f (a + 2h) + f (a + 3h) + . . . + f (b $-$ h) + 0.5f (b)] .
\end{flushleft}





(14)





\begin{flushleft}
Para escrever um c\'{o}digo com esse m\'{e}todo, \'{e} mais eficiente calcularmos separadamente a contribui\c{c}\~{a}o dos extremos,
\end{flushleft}


\begin{flushleft}
dada por 0.5h (f (a) + f (b)), e realizarmos um la\c{c}o para calcularmos as contribui\c{c}\~{o}es restantes das parti\c{c}\~{o}es internas.
\end{flushleft}


\begin{flushleft}
Podemos agora considerar uma aproxima\c{c}\~{a}o quadr\'{a}tica para f (x) ao redor de x = u
\end{flushleft}


?


?


?


?


\begin{flushleft}
1 f (u + h) $-$ 2f (u) + f (u $-$ h)
\end{flushleft}


\begin{flushleft}
f (u + h) $-$ f (u)
\end{flushleft}


2


\begin{flushleft}
(x $-$ u) +
\end{flushleft}


\begin{flushleft}
f (x) $\approx$ f (u) +
\end{flushleft}


\begin{flushleft}
(x $-$ u) ,
\end{flushleft}


\begin{flushleft}
h
\end{flushleft}


2


\begin{flushleft}
h2
\end{flushleft}


\begin{flushleft}
onde utilizamos a Eq. 8. Nesse caso, precisamos de tr\^{e}s pontos em nosso intervalo e a integral que avaliaremos \'{e}
\end{flushleft}


\begin{flushleft}
ˆ u+h
\end{flushleft}


1


\begin{flushleft}
h
\end{flushleft}


(15)


\begin{flushleft}
I=
\end{flushleft}


\begin{flushleft}
f (x) dx = f (u) 2h + (f (u + h) $-$ 2f (u) + f (u $-$ h)) = (f (u + h) + 4f (u) + f (u $-$ h)) ,
\end{flushleft}


3


3


\begin{flushleft}
u$-$h
\end{flushleft}


?


\begin{flushleft}
que \'{e} conhecida como regra de Simpson. Seu erro global \'{e} O h4 . A integral no intervalo [a, b] \'{e} ent\~{a}o dada por
\end{flushleft}


\begin{flushleft}
h
\end{flushleft}


\begin{flushleft}
[f (a) + 4f (a + h) + 2f (a + 2h) + 4f (a + 3h) + 2f (a + 4h) . . . + f (b)] .
\end{flushleft}


3


\begin{flushleft}
Novamente, para a confec\c{c}\~{a}o do c\'{o}digo para esse m\'{e}todo, \'{e} mais eficiente calcularmos separadamente a contribui\c{c}\~{a}o
\end{flushleft}


\begin{flushleft}
dos extremos, agora dada por h3 (f (a) + f (b)) , e realizarmos um la\c{c}o para calcularmos as contribui\c{c}\~{o}es restantes das
\end{flushleft}


\begin{flushleft}
parti\c{c}\~{o}es internas. Note, contudo, que nesse os termos de ordem par e \'{i}mpar possuem pesos diferentes.
\end{flushleft}


\begin{flushleft}
Podemos continuar esse processo e considerar cada vez mais pontos e polin\^{o}mios de grau superior como aproxima\c{c}\~{o}es
\end{flushleft}


\begin{flushleft}
para f (x). Contudo, na pr\'{a}tica a s\'{e}rie de Taylor \'{e} apenas localmente convergente e portanto n\~{a}o \'{e} imediato que um
\end{flushleft}


\begin{flushleft}
polin\^{o}mio de ordem maior se traduza em uma melhor converg\^{e}ncia. As duas aproxima\c{c}\~{o}es descritas acima formam
\end{flushleft}


\begin{flushleft}
a base de m\'{e}todos bem robusto para avaliarmos integrais mesmo de fun\c{c}\~{o}es n\~{a}o muito suaves. O segredo do sucesso
\end{flushleft}


\begin{flushleft}
do m\'{e}todo est\'{a} em variarmos o valor das parti\c{c}\~{o}es N at\'{e} em obtermos a precis\~{a}o desejada.
\end{flushleft}


\begin{flushleft}
Baseado nessa discuss\~{a}o, considere agora a seguinte fun\c{c}\~{a}o
\end{flushleft}


\begin{flushleft}
x
\end{flushleft}


\begin{flushleft}
f (x) = e2x cos .
\end{flushleft}


(16)


4


´1


\begin{flushleft}
(a) Escreva um programa FORTRAN que calcule I = 0 f (x) dx tanto pelo m\'{e}todo do trap\'{e}zio quanto pelo de
\end{flushleft}


\begin{flushleft}
Simpson para diferentes valores da parti\c{c}\~{a}o h = 1/N .
\end{flushleft}


\begin{flushleft}
(b) Preencha a tabela II com o valor absoluto dos desvios em rela\c{c}\~{a}o ao resultado exato |$\epsilon$|. Aponte o valor \'{o}timo
\end{flushleft}


\begin{flushleft}
de h em cada um dos casos e discuta seus resultados.
\end{flushleft}


\begin{flushleft}
(c) Fa\c{c}a um gr\'{a}fico de log10 |$\epsilon$| × log10 h para ambos os m\'{e}todos. Verifique que a ordem da converg\^{e}ncia das
\end{flushleft}


\begin{flushleft}
aproxima\c{c}\~{o}es coincide com aquela esperada teoricamente e discuta seus resultados.
\end{flushleft}


\begin{flushleft}
IS $\approx$
\end{flushleft}





\newpage
5





\begin{flushleft}
Figura 2: Exemplo de um console do Xmgrace com os resultados de uma regress\~{a}o linear.
\end{flushleft}





\begin{flushleft}
Breve discuss\~{a}o sobre a execu\c{c}\~{a}o dos problemas
\end{flushleft}





\begin{flushleft}
Nesses problemas \'{e} pedido que voc\^{e}s fa\c{c}am um gr\'{a}fico log10 |$\epsilon$| × log10 h. A ideia aqui \'{e} o erro |$\epsilon$| depende de uma
\end{flushleft}


\begin{flushleft}
pot\^{e}ncia de h
\end{flushleft}


\begin{flushleft}
|$\epsilon$| = Ah$\alpha$ ,
\end{flushleft}





(17)





\begin{flushleft}
onde A \'{e} uma constante num\'{e}rica sem import\^{a}ncia para nossa discuss\~{a}o e $\alpha$ \'{e} o expoente no qual estamos interessados.
\end{flushleft}


\begin{flushleft}
Por exemplo quando dizemos que a converg\^{e}ncia de um m\'{e}todo \'{e} linear, quer dizer que $\alpha$ = 1. Uma converg\^{e}ncia
\end{flushleft}


\begin{flushleft}
quadr\'{a}tica significa $\alpha$ = 2 e assim por diante.
\end{flushleft}


\begin{flushleft}
Em situa\c{c}\~{o}es nas quais temos uma depend\^{e}ncia do tipo lei de pot\^{e}ncia, \'{e} conveniente aplicarmos o log dos dois
\end{flushleft}


\begin{flushleft}
lados para extrairmos o valor do expoente em uma escala linear. Pela Eq. (17) temos ent\~{a}o que
\end{flushleft}


\begin{flushleft}
log10 |$\epsilon$| = log10 Ah$\alpha$ ,
\end{flushleft}


\begin{flushleft}
= log10 A + $\alpha$ log10 h.
\end{flushleft}





(18)





\begin{flushleft}
Ou seja, ao gerarmos a curva log10 |$\epsilon$| × log10 h conclu\'{i}mos que o coeficiente angular da reta d\'{a} diretamente o expoente
\end{flushleft}


\begin{flushleft}
$\alpha$. Esse \'{e} o n\'{u}mero no qual estamos interessados para estudarmos a converg\^{e}ncia de cada um dos m\'{e}todos (o
\end{flushleft}


\begin{flushleft}
coeficiente linear pode ser ignorado nessa discuss\~{a}o).
\end{flushleft}


\begin{flushleft}
Para calcular o log na base 10, basta utilizarem a fun\c{c}\~{a}o DLOG10(). Voc\^{e}s dever\~{a}o ent\~{a}o gerar um gr\'{a}fico no
\end{flushleft}


\begin{flushleft}
Xmgrace com essa curva. O pr\'{o}ximo passo \'{e} realizar uma regress\~{a}o linear para calcularmos $\alpha$.
\end{flushleft}


\begin{flushleft}
$\bullet$ Para ler o arquivo foo.dat e grafic\'{a}-lo fa\c{c}a: Data -$>$ Import -$>$ ASCII e escolha foo.dat.
\end{flushleft}


\begin{flushleft}
$\bullet$ Para realizar a regress\~{a}o linear fa\c{c}a: Data -$>$ Transformation -$>$ Regression. Isso vai gerar uma janela. Basta
\end{flushleft}


\begin{flushleft}
clicar em Accept, pois a op\c{c}\~{a}o padr\~{a}o \'{e} a regress\~{a}o linear. Pronto, a curva j\'{a} \'{e} gerada automaticamente
\end{flushleft}


\begin{flushleft}
$\bullet$ Para visualizar a express\~{a}o anal\'{i}tica da curva, v\'{a} em Window -$>$ Console, veja um exemplo na Fig. 2
\end{flushleft}


\begin{flushleft}
$\bullet$ A express\~{a}o para a reta aparece, no exemplo na Fig. 2, como y = 1.2694 + 1.0135 ∗ x. O valor do coeficiente
\end{flushleft}


\begin{flushleft}
angular \'{e} ent\~{a}o 1.01350 (2) ou 1.01350 $\pm$ 2, j\'{a} com o erro. O erro pode ser encontrado na linha {``}Standard error
\end{flushleft}


\begin{flushleft}
of coefficient'' e nesse exemplo \'{e} dado por 2.46 × 10$-$5 .
\end{flushleft}





\newpage



\end{document}
